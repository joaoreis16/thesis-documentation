\chapter{Introduction}
\label{chapter:Introduction}

%\begin{introduction}
%
% 
%
%\end{introduction}

% falar sobre a historia de fazer estudos distribuidos...
% um investigador, várias BDs medicas, e existe um problema que as vezes ha falta de doentes numa BD. Surgiu a ideia de usarem multiplas DBs. Isto leva a outros problemas de iteroperabilidade. OMOP CDM e OHDSI.....

The continuous quest for medical answers and advancements in clinical research, combined with the diversity of medical databases, has sparked complex challenges for researchers. A recurring issue is the scarcity of specific medical data that are the focus of a study, such as cases of patients with rare diseases. In this regard, a promising strategy has emerged, which consists of integrating multiple and diverse medical databases.

However, the implementation of this strategy is not free of obstacles, with the issue of heterogeneity between databases being a prominent challenge. In other words, databases contain different types, formats and/or sources, which are often not compatible with each other. The existence of these diverse data is not very effective, as they cannot be easily shared or integrated with other data.

It is in this context that the {\omop} and the {\ohdsi} initiative have emerged as good solutions to the problem of heterogeneity and interoperability among clinical medical data. Generally speaking, {\omop} is a common data model that establishes a universal standard for representing patient clinical information, allowing for interoperability among disparate databases. The {\ohdsi} initiative is, in turn, an international collaboration composed of researchers and scientists committed to the mission of developing analytical, open-source solutions for an extensive network of medical databases, following systematic analysis of this heterogeneous data.

With the assistance of {\omop} and {\ohdsi}, the challenges of data interoperability are overcome, enabling the discovery of crucial insights for advancing and improving medical studies. Sharing this data presents numerous advantages for researchers, including promoting new fields of study and a significant increase in the impact and recognition of research results.


\section{Motivation}
% 1º Problema: como econtrar as BDs? Alguem resolveu isso atraves de catalogues.... EHDEN
% 2º Problema: como é que eu escolho as bd de interesse nesses catalogues? Alguem criou uma forma de caracterizar essa BDs usando informações estatisticas e agregadas (Networkdashboards)
% 3º Problema: Ok isso é giro, mas algumas bd/conceitos são dispensos sendo dificil chegar à query que se pretende para o estudo. e este é o desafio

The search for data sources of interest for a researcher’s study can be complex due to the large number of databases in the community. To face this challenge, some of these databases are grouped into database catalogues. This strategy consists of characterizing the data by aggregating data and metadata.

The {\ehden} portal is an excellent example of a platform that provides a catalogue of medical databases from across Europe. It is a centralized repository that facilitates the discovery of relevant data sources for researchers.

Despite the assistance provided by the catalogue offered by {\ehden}, identifying the most suitable databases for a specific study remains a challenge. Thus, to facilitate search in the catalogue, Networkdashboards has emerged, offering statistical and aggregated information about the databases available on the {\ehden} network. With this tool, researchers can filter the most suitable databases for their research needs and make more informed decisions.

Even with all this help, choosing the most appropriate databases is difficult and time-consuming, making it difficult to achieve the ideal search desired for the study. The challenge to be addressed is to assist a medical researcher in reaching the ideal search based on the protocol and parameters of their study.


\section{Objectives}

% How can a conversational query builder support medical researchers when defining a study protocol?
% Goals:
% 1. study of state-of-the-art
% 2. developed a chat-like search engine to help discover the best databases for a study
% 3. enhance the engine to collect additional information to provide a query as outcome.


The main objective of this work is to develop a conversational query builder to help medical researchers define their study objectives. To achieve this objective, the present dissertation seeks to answer the following research question:

\begin{quote}
    \small\textit{How can a conversational query builder support medical researchers when defining a study protocol?}
\end{quote}

To answer this question, the work can be addressed by focusing on different aspects, namely by dividing it into three stages:

\begin{enumerate}
    \item Study of state-of-the-art, namely: i) Information retrieval systems, ii) Generative-based conversational virtual agents, and iii) Interactive query builder;
    \item Developed a chat-like search engine to help discover the best databases for a study;
    \item Enhance the engine to collect additional information to provide a query as outcome. 
\end{enumerate}



% \section{Dissertation outline}

% Que resume a estrutura do documento.
