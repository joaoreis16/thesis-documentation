\chapter{Conclusions}
\label{chapter:Conclusions}

% sumário dos findings, contribuições chave, bm25 fucniona ou não, chat é bom nisto ou não, future work, no trabalho deu origem a três artigos

%% REFERIR OS OBJETIVOS E DIZER QUE FORAM CUMPRIDOS

% 1) Study of state-of-the-art: i) methodologies to build a conversational user interface, ii) procedures to retrieve the databases of most interest, and iii) explore the definition of a study protocol;
% 2) Developed a chat-like search engine to help discover the best databases for a study;
% 3) Enhance the engine to collect additional information to provide a query as an outcome. 


% this dissertation proposes / research question e a resposta
This dissertation proposes developing a conversational system to assist medical researchers in defining their study objectives. Answering the research question, this dissertation provides a personalized and efficient method for biomedical databases discovery and definition of a study protocol. Researchers can interact with the system conversationally, outlining their requirements and receiving customized recommendations.

% objetvos concluidos
The research question guided to accomplish three objectives. The literature review on conversational interfaces, {\nlp}, and {\ir} was crucial to develop a chat-like search engine to help discover the best databases for a study and enhance the engine to support the cohort definition for observational studies. The successful implementation of these objectives not only simplifies the process of identifying the most relevant databases, but also helps in the definition of study protocols process, empowering researchers, regardless of their prior experience with database search tools and query builders.

% resultados de implementação
Regarding implementation, it is possible to extract some conclusions about the choices, strategy, and areas. About BM25, it shows promising results, but creating an annotated dataset may enable us to identify how to improve these results or other powerful {\ir} techniques. To add the {\llm} into the system, FlowiseAI shows to be an excellent tool for the development of {\llm}-based applications. However, it is limited due to this project's complexity, specifically the query builder phase. So, the option to integrate the {\llm} is a Python-based backend developed for this project.

% resultados do sistema
In conclusion, the introduction of this conversational search tool addresses the significant challenge of navigating a growing repository of medical databases. This system enables users to engage in natural language dialogue, significantly reducing the time and complexity traditionally associated with identifying suitable databases for research studies. Moreover, the system reflects a deep understanding of the needs within the medical research community for reliable, up-to-date, and accessible data.

Also, introducing a conversational query builder is innovative due to the limited existence of a query builder that works in a conversational manner. This feature of the system not only saves time for medical researchers but also saves the researcher from the complexity of defining a cohort or a concept set in the current tool, ATLAS.

% future work
However, some work remains to be done in the future, namely to enhance the query builder feature in order to provide the full cohort definition. Due to the complexity of a cohort definition, the current system only attempts the concept set definition and the observation window, which are parts of the cohort. More work is needed to enhance this system, such as improving the user interface, enhancing the {\ir} methods to produce better results, and creating synthetic labeled dataset to test and validate the {\ir} methods.

% final
The proposed system addresses some real and meaningful challenges in the medical research field, essentially the complexity of finding the best databases in a vast catalogue and defining a study protocol in the ATLAS platform. It led to the creation of three scientific papers that contributed to the advancement of the field. This system has the potential to be very useful for medical researchers and play a significant role in driving technological progress.
